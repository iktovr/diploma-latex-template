\section{ЗАДАНИЕ СПИСКА ТЕРМИНОВ, СОКРАЩЕНИЙ И ОПРЕДЕЛЕНИЙ}

Определения необходимо записывать в файл \texttt{terms.tex}, сокращения --- в \texttt{glossary.tex}.
Файлы необходимо подключать до начала документа, при помощи \texttt{\textbackslash input}.

Обратите внимание, что термины и сокращения необходимо приводить в алфавитном порядке, пока что это можно сделать только вручную.

Для того чтобы списки подключились в качестве части отчёта, 
нужно использовать команды \texttt{\textbackslash termsanddefenitions} и \texttt{\textbackslash listofabbreviations} 
сразу после команды \texttt{\textbackslash tableofcontents}.

Определения и сокращения можно помещать в один структурный элемент, для этого оформляйте их как термины (файл \texttt{terms.tex}). Для создания списка используйте команду \texttt{\textbackslash termsandabbreviations}.
